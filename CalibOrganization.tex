% Calibration organization

The calibrations all share a similar format.
\begin{itemize}
\item The PixelSupervisor controls a loop over 'events' and coordinates
      the activity of the FED, FEC, TKFEC, and TTC supervisors.
\item Each calibration needs to do some initialization before processing
      the event data.
\item During event processing data is acquired and processed.
\item After all event data is taken it is analyzed and the results are
      saved.
\end{itemize}
This section will describe a {\it proposal} for how to formalize this process.
A first step in this this direction was taken when the calibration classes
were broken out from the the PixelSupervisor and the PixelFEDSupervisor.
This first step was almost trivial as it just copied the code out to
separate classes. There are a few more changes we should make that will
allow us to run these jobs in workloops. Among other things, this will 
allow us to look
at the progress of the calibrations from the web browser.

\subsection{PixelSupervisor calibration code}

We change the {\tt PixelCalibrationBase} class to have
the interface below.
\begin{verbatim}
class PixelCalibrationBase : public PixelSupervisorConfiguration, 
	    public SOAPCommander {
 public:

  PixelCalibrationBase( const PixelSupervisorConfiguration &,
			   const SOAPCommander& soapCommander  );

  virtual ~PixelCalibrationBase(){};

  virtual void beginCalibration();

  virtual bool calibrationEvent();

  virtual void endCalibration();

 protected:
  
  private:
  
  // PixelCalibrationBase Constructor
  PixelCalibrationBase(const SOAPCommander& soapCommander);

};
\end{verbatim}

The method {\tt begin()} should do any initialization that 
is needed before processing any triggers. The method {\tt event()}
is called repeatedly until it returns true, this indicates that
the calibration has come to an end.
Then the {\tt end()}
method is invoked to perform any analysis of the acquired data.


More details on what these methods needs to do will be
discussed below.


\subsection{PixelFEDSupervisor calibration code}

Similarly, I suggest a more general structure for the calibration 
code that runs in the PixelFEDSupervisor.

\begin{verbatim}
class PixelFEDCalibrationBase : public PixelFEDSupervisorConfiguration, 
	    public SOAPCommander {
 public:

  PixelFEDCalibrationBase( const PixelFEDSupervisorConfiguration &,
			   const SOAPCommander& soapCommander );

  virtual ~PixelFEDCalibrationBase(){};

  virtual xoap::MessageReference beginCalibration(xoap::MessageReference msg);

  virtual xoap::MessageReference calibrationEvent(xoap::MessageReference msg);

  virtual xoap::MessageReference endCalibration(xoap::MessageReference msg);

  //Should really not be here....
  unsigned int TransparentDataStart (unsigned long *buffer);


 protected:

  //SOAPCommander* theSOAPCommander_;

 private:

  // PixelCalibrationBase Constructor should neve be called
  PixelFEDCalibrationBase(const SOAPCommander& soapCommander);

};
\end{verbatim}

Again the three methods {\tt beginCalibration()}, 
{\tt calibrationEvent()}, and {\tt endCalibration()}
corresponds to the pre calibration data taking, processing of trigger
and the post data taking analysis respectively. I suggest that
these methods are bound to soap messages that invokes
the implemented methods.

In particular we should move the code that sets the mode
and control registers to the {\tt beginCalibration()} method.
This will allow us to move out code from the PixelFEDSupervisor
that is specific to different calibrations. {\it Or we split
out the mode and control register settings to a new configuration
object.}

\subsection{Implementation of algorithms}

During the configuration step the {\tt PixelSupervisor} and
{\tt PixelFEDSupervisor}
instantiates the Calibration class for the calibration objects.
When we go to Run the PixelSupervisor invokes the {\tt beginCalibration()}
method. This method is responsible for sending the {\tt beginCalibration()}
message to the PixelFEDSupervisors. 

After this the PixelSupervisor should execute the {\tt calibrationEvent()}
in a workloop until it has completed. Again, the code executed in the
PixelFEDSupervisor is responsible for calling the appropriate 
supervisors. This gives the maximal flexibility in the calibration
algorithm.

Last, the {\tt endCalibration} method in the {\tt PixelSupervisor} is
invoked to complete the analysis of the collected data and report
the results.

In addition it would be nice to take out the code from the PixelSupervisor
and the PixelFEDSupervisor that select the calibration type. One
way of doing this is to create a 'factory' for calibrations in the 
PixelCalibrations package. This factory would accept a string from
the calibration mode and create the calibration object and return
it to the PixelSupervisor and the PixelFEDSupervisor. Besides
simplifying the code in these supervisors (they will not need to \#include
all the calibrations header files) it will also separate out dependencies.
