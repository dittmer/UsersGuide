
% Procedures to follow after replacing hardware

This section documents the procedures that are to be followed
when off detector hardware components are replaced that 
affects the online software and calibrations. {\it This 
section is not complete!}

\subsection{FED board}

As the FED decodes the analog levels it will be required to
redo several calibrations to check the optical connections
after replacing a FED. The FED firmware version has to
be validated.
The following calibrations should be run
\begin{itemize}
\item FED baseline
\item AOH gain
\item AOH bias
\item FED baseline (it is possible that the previous three steps
          has to be repeated.)
\item FEC clock and phase
\item TBM UB
\item ROC UB
\item Address level
\end{itemize}

\subsection{Pixel FEC including mFECs}

 No calibrations should be affected by replacing a FEC board
or a mFEC --- in principle. However, we have seen that the 
size of the window for good communication changes a little
between different mFECs. This seems mostly to affect the 
return data and this check is now disabled for both FPix
and BPix. It is probably prudent to rerun a Delay25 calibration
to check that after reconnecting the fibers we have the expected 
performance.

Need to verify that the right firmware is installed. I assume
that we manage to connect the fibers in the right slot.

\subsection{TKFEC}

 No calibrations should be affected by replacing a TKFEC board.

Need to verify that the right firmware is installed. I assume
that we manage to connect the fibers in the right slot.

