% Configuration Objects

This section describes the implementation
of the configuration objects for the 
pixel online system. The interface for accessing this data 
was discussed in the previous section. 

\subsection{Introduction}

Key goals for the design of the configuration
data object include:
\begin{itemize}
\item The configuration has to be fast and reliable.
      The fewer components, read pieces of software, 
      involved in the configuration step the more
      likely the system is to work reliably.
\item The data volume should be small. I.e. the data
      has to be packed in an efficient way.
\item Want to optimize the database access by accessing
      relatively few, but large, objects. 
\item Computers are good at manipulating data that is in
      memory, so the actual commands that are sent to 
      the hardware can be built on the fly -- assuming that
      all information to do this is accessible in memory.
\item The data volume for the whole pixel system is ${\cal O}(100\ {\rm MB})$
      so holding this in memory in a single computer should
      not be an issue. In fact this will be spread over
      more than one computer as the FECs are in more than
      one crate.
\end{itemize}

The following classes are used to configure the online pixel
applications:

\vskip 0.5cm
\noindent
{\bf PixelTrimAllPixels}: This class stores the trim bits for the
                          ROCs on one module. The trims are stored
                          for each pixel.

\vskip 0.5cm
\noindent
{\bf PixelMaskAllPixels}: This class stores the mask bits for the
                          ROCS on one module. The masks are stored
                          for each pixel.

\vskip 0.5cm
\noindent
{\bf PixelDACSettings}: This class stores the DAC settings for all
                        ROCs on one module. 

\vskip 0.5cm
\noindent
{\bf PixelTBMSettings}: This class stores the TBM settings for one
                        TBM.

\vskip 0.5cm
\noindent
{\bf PixelNameTranslation}: This class translates from the pixel 
                            naming scheme documents names of ROCs to
                            the hardware addresses used by both the
                            FEC and the FED to identify a ROC.

\vskip 0.5cm
\noindent
{\bf PixelDetectorConfig}: This class lists the modules used in a 
                           configuration. The utility of this class
                           is that it allows one to only use a small
                           subset of the detector without having to 
                           create a new name translation.

\vskip 0.5cm
\noindent
{\bf PixelROCStatus}: This class keeps track of the status of 
                      ROCs. The default assumption is that a ROC
                      is working and this object allows us to 
                      list ROCs that are not working, or that we 
                      want to turn off.

\vskip 0.5cm
\noindent
{\bf PixelFECConfig}: This class lists the pixel FECs that are used.

\vskip 0.5cm
\noindent
{\bf PixelTKFECConfig}: This class lists the tracker FECs that are used.

\vskip 0.5cm
\noindent
{\bf PixelFEDConfig}: This class lists the pixel FEDs that are used in
                      the configuration.

\vskip 0.5cm
\noindent
{\bf PixelFEDCard}: This class stores the settings for one FED board.

\vskip 0.5cm
\noindent
{\bf PixelPortCard}: This class stores the settings on a portcard, e.g.
                     the delay25 settings and AOH settings.

\vskip 0.5cm
\noindent
{\bf PixelPortcardMap}: This class maps the AOH channels on portcards
                        to the FED channels.


\vskip 0.5cm
\noindent
{\bf PixelLTCConfiguration}: This class holds the configuration for the 
                             pixel LTC module.
\vskip 0.5cm
\noindent
{\bf PixelTTCciConfiguration}: This class holds the configuration for the 
                               pixel TTCci module.

\vskip 0.5cm
\noindent
{\bf PixelCalibConfiguration}: This class specifies how the calibrations
                               are executed. 

\vskip 0.5cm
\noindent
{\bf PixelDelay25Configuration}: This class specifies how the delay25 
                                 calibration is executed.

\vskip 0.5cm


With the exception of {\tt PixelCalibConfiguration} and 
{\tt PixelDelay25Configuration} the classes above are used to build the
configuration of the hardware. The last two classes are used to 
configure the software applications to perform a given 
calibration.


The following sections describe what is implemented, starting with
a common base class for all configuration objects and then a
brief discussion of all classes currently implemented.

All configuration objects derive from a common base class called
{\tt PixelConfigBase}.

\begin{verbatim}
class PixelConfigBase {

 public:

    //A few things that you should provide
    //description : purpose of this object
    //creator : who created this configuration object
    //date : time/date of creation (should probably not be
    //       a string, but I have no idea what CMS uses.
    PixelConfigBase(std::string description,
		    std::string creator,
                    std::string date);

    virtual ~PixelConfigBase(){}

    std::string description();
    std::string creator();
    std::string date();

    //Interface to write out data to ascii file
    virtual void writeASCII(std::string dir="") const = 0;

 private:

    std::string description_;
    std::string creator_;
    std::string date_;
     

};
\end{verbatim}

This class is intended to provide a common interface. Currently it simply
holds information about when the objects were created. 


\subsection{Trim and mask bits}

The trim and mask bits need to be set for each pixel. This
makes these the largest configuration objects. In terms
of configuration of the ROC, the mask and trim bits are loaded
together. However, as the mask bits are used offline we will
store the trim and mask bits in different objects 
as this will reduce the data that needs to be downloaded
to access the mask bits offline. Also, this saves space
as compared to storing the trim and mask bit for each 
channel as one byte. It is also likely that we would
like to update the mask bits without changing the trim bits.
This would happen for example when we discover a hot pixel
and want to mask it off. Being able to change only the
mask bits would then be useful.

For both the mask and trim bits we allow for different implementations.
For example, we can have one implementation that allows us
to use different settings for each pixel, or we can have a different
configuration that uses the same settings for all pixels. The code
handles this transparently by using inheritance. 

For the mask bits the base class looks like:

\begin{verbatim}
class PixelMaskBase: public PixelConfigBase {

 public:

    PixelMaskBase(std::string description, 
		  std::string creator,
		  std::string date);

    virtual ~PixelMaskBase();

    void setOverride(PixelMaskOverrideBase*);

    virtual const PixelROCMaskBits& getMaskBits(int ROCId) const =0;

    virtual void writeBinary(std::string filename) const =0;

    virtual void writeASCII(std::string filename) const =0;

    friend std::ostream& operator<<(std::ostream& s, const PixelMaskBase& mask);

 private:

    //Hold pointer to the mask override information.
    PixelMaskOverrideBase* maskOverride_;


};
\end{verbatim}

The concrete implementation that implements mask bits for each
channel looks like:

\begin{verbatim}
class PixelMaskAllPixels: public PixelMaskBase {

 public:

    PixelMaskAllPixels(std::string filename);

    void writeBinary(std::string filename) const;

    void writeASCII(std::string filename) const;

    const PixelROCMaskBits& getMaskBits(int ROCId) const;

 private:

    std::vector<PixelROCMaskBits> maskbits_;  
 
};

\end{verbatim}

The file format that we use looks like:
\begin{verbatim}
ROC:     FPix_BpI_D1_BLD1_PNL1_PLQ2_ROC1
col00:   11110000000000000000000000000000000000000000000000000000000000000000000000000000
col01:   00000000000000000000000000000000000000000000000000000000000000000000000000000000
col02:   00000000000000000000000000000000000000000000000000000000000000000000000000000000
.
.
.
col49:   00000000000000000000000000000000000000000000000000000000000000000000000000000000
col50:   00000000000000000000000000000000000000000000000000000000000000000000000000000000
col51:   00000000000000000000000000000000000000000000000000000000000000000000000000000000
ROC:     FPix_BpI_D1_BLD1_PNL1_PLQ2_ROC2
col00:   10000000000000000000000000000000000000000000000000000000000000000000000000000000
col01:   20000000000000000000000000000000000000000000000000000000000000000000000000000000
col02:   30000000000000000000000000000000000000000000000000000000000000000000000000000000
.
.
.
\end{verbatim}
Where the file contains the data for each of the ROCs in a module.
Within each ROC the trim bits are listed for each column by the
value if the trim bit as one hexadecimal character from 0 to F.


Similarly for the trim bits we have the base class:

\begin{verbatim}
class PixelTrimBase: public PixelConfigBase {

 public:

    PixelTrimBase(std::string description, 
		  std::string creator,
		  std::string date);

    virtual ~PixelTrimBase();
    
    void setOverride(PixelTrimOverrideBase* trimOverride);

    //Build the commands needed to configure ROCs
    //on control link

    virtual void generateConfiguration(PixelFECConfigInterface* pixelFEC,
				       PixelNameTranslation* trans,
				       const PixelMaskBase& pixelMask) const =0;
    virtual void writeBinary(std::string filename) const =0;

    virtual void writeASCII(std::string filename) const =0;

    virtual PixelROCTrimBits getTrimBits(int ROCId) const =0;

    friend std::ostream& operator<<(std::ostream& s, const PixelTrimBase& mask);


 private:

    PixelTrimOverrideBase* trimOverride_;

};
\end{verbatim}

And the concrete implementation looks like:

\begin{verbatim}
class PixelTrimAllPixels: public PixelTrimBase {

 public:

    PixelTrimAllPixels(std::string filename);

    //Build the commands needed to configure ROCs
    //on control link

    void generateConfiguration(PixelFECConfigInterface* pixelFEC,
			       PixelNameTranslation* trans,
			       const PixelMaskBase& pixelMask) const;

    void writeBinary(std::string filename) const;

    void writeASCII(std::string filename) const;

    PixelROCTrimBits getTrimBits(int ROCId) const;

 private:

    std::vector<std::string> rocname_;
    std::vector<PixelROCTrimBits> trimbits_;

};
\end{verbatim}

We use basically the same format for the mask bits as we used for the
trim bits:

\begin{verbatim}
ROC:    FPix_BpI_D1_BLD1_PNL1_PLQ2_ROC1
col00:  11111111111111111111111111111111111111111111111111111111111111111111111110000000
col01:  11111111111111111111111111111111111111111111111111111111111111111111111111111111
col02:  11111111111111111111111111111111111111111111111111111111111111111111111111111111
.
.
.
col49:  11111111111111111111111111111111111111111111111111111111111111111111111111111111
col50:  11111111111111111111111111111111111111111111111111111111111111111111111111111111
col51:  11111111111111111111111111111111111111111111111111111111111111111111111111111111
ROC:    FPix_BpI_D1_BLD1_PNL1_PLQ2_ROC2
col00:  11111111111111111111111111111111111111111111111111111111111111111111111111111111
col01:  11111111111111111111111111111111111111111111111111111111111111111111111111111111
col02:  11111111111111111111111111111111111111111111111111111111111111111111111111111111
.
.
.
\end{verbatim}
Here the mask bits are either 0 or 1 for each pixel.

\subsection{ROC DACs}

The DAC settings for each readout chip are stored in the class

\begin{verbatim}
class PixelDACSettings: public PixelConfigBase {

 public:

    PixelDACSettings(std::string filename);

    PixelROCDACSettings getDACSettings(int ROCId) const;

    //Generate the DAC settings
    void generateConfiguration(PixelFECConfigInterface* pixelFEC,
	                       PixelNameTranslation* trans) const; 

    void writeBinary(std::string filename) const;

    void writeASCII(std::string filename) const;

    friend std::ostream& operator<<(std::ostream& s, const PixelDACSettings& mask);

 private:

    std::vector<PixelROCDACSettings> dacsettings_;   
 
};

\end{verbatim}
The format for the DAC settings in the ASCII format is given by
\begin{verbatim}
ROC:           FPix_BpI_D1_BLD1_PNL1_PLQ2_ROC1
Vdd:           6
Vana:          140
Vsf:           128
Vcomp:         15
Vleak_comp:    0
VrgPr:         0
VallPr:        35
VrgSh:         0
VwllSh:        35
VHldDel:       90
Vtrim:         29
VthrComp:      70
VIbias_Bus:    30
Vbias_sf:      6
VoffsetOp:     30
VIbiasOp:      115
VoffsetRO:     100
VIon:          115
VIbias_PH:     90
VIbias_DAC:    100
VIbias_ROC:    160
VIColOr:       99
Vnpix:         0
VsumCol:       0
VCal:          80
CalDel:        90
WBC:           120
ChipContReg:   0
ROC:           FPix_BpI_D1_BLD1_PNL1_PLQ2_ROC2
Vdd:           6
Vana:          140
Vsf:           128
.
.
.
\end{verbatim}
Where the file contains the DAC settings for all the ROCs in a given module.

\subsection{PixelDetectorConfig}

Specifies which components of the detector are used in the configuration.
The level of configurability is the module (or in FPIX language 
the panel). I.e. the
components that are controlled by one TBM.

The file format that we use to specify this format looks like:
\begin{verbatim}
FPix_BmI_D1_BLD1_PNL1
FPix_BmI_D1_BLD1_PNL2
.
.
.
\end{verbatim}

The format above is the 'old' format. After discussions with the
database GUI developers we have decided to make this object
specify the ROC and their status. All ROCs on a module have
to be listed. Otherwise there is an internal inconsistency in
the configuration. In addition to listing the ROC one can
specify a status of the ROC. An example of the file is
given below
\begin{verbatim}
Rocs:
FPix_BpI_D1_BLD1_PNL1_PLQ2_ROC0 
FPix_BpI_D1_BLD1_PNL1_PLQ2_ROC1 
FPix_BpI_D1_BLD1_PNL1_PLQ2_ROC2 off
FPix_BpI_D1_BLD1_PNL1_PLQ2_ROC3 
FPix_BpI_D1_BLD1_PNL1_PLQ2_ROC4 noAnalogSignal
FPix_BpI_D1_BLD1_PNL1_PLQ2_ROC5 
FPix_BpI_D1_BLD1_PNL1_PLQ2_ROC6 off noHits
FPix_BpI_D1_BLD1_PNL1_PLQ2_ROC7 
FPix_BpI_D1_BLD1_PNL1_PLQ2_ROC8 
FPix_BpI_D1_BLD1_PNL1_PLQ2_ROC9 
\end{verbatim}
The status words are explained in more detail in Sec.~\ref{sec:rocstatus}.

The class {\tt PixelConfigurationVerifier} checks the internal
consistency of the detector configuration. For instance, it checks
that if any ROC on a FED channel is marked with {\tt noAnalogSignal},
then the entire FED channel is similarly marked. Also, it ensures the
consistency of the FED card with the detector configuration. If an
entire FED channel is marked as {\tt noAnalogSignal}, then the
corresponding FED channel is automatically disabled. Similarly, if a
FED channel is disabled in the FED card, but enabled in the detector
configuration, then the FED card is dynamically modified to conform to
the detector configuration. In this way the the detector configuration
is the ``master'' flag for what parts of the detector are included in
the configuration.

{\em We should check if the detector configuration currently controls
which portcard devices are initialized. Ideally if a portcard is not
used, then it should not be initialized during the configuration.}


\subsection{PixelROCStatus}
\label{sec:rocstatus}
The {\tt PixelROCStatus} class is used to store the status
of ROCs. The default assumption is that a ROC is working and
is on. However, we are likely to have problems with some ROCs
given the number of components we have. This structure should
allow us to add new failure modes as we discover new 
problems. An example of the data would look like
\begin{verbatim}
FPix_BpI_D1_BLD1_PNL1_PLQ2_ROC2 noHits
FPix_BpI_D1_BLD1_PNL1_PLQ2_ROC7 off
FPix_BpI_D1_BLD1_PNL1_PLQ2_ROC8 off noHits
\end{verbatim}

The different status flags that we can set are:
\begin{itemize}
\item {\tt noHits} indicates that we can not generate hits on the ROC.
      For example this means that the ROC can not be calibrated
      e.g. for address levels. However, it is not preventing us from
      doing the UB equalization. In principle this flag should be
      handled on a calibration-by-calibration basis, but at the moment
      it is ignored.
\item {\tt off} indicates that the ROC is disabled via a control bit on the ROC.
                It is not used in any calibration and will not generate hits.
                However, even it a ROC is {\tt off} it will be
                configured. Presently the use of this flag is not implemented (it is ignored by the code).
\item {\tt noInit} indicates that the ROCs in a module should not be included in the configuration.
		This is implemented.
\item {\tt noAnalogSignal}  indicates that the ROC can configured, but that something in the analog readout is broken. The ROC is included in the configuration but excluded from calibrations. This is implemented, and the corresponding FED channel is automatically disabled.
\end{itemize}
With the file-based configuration, we can set more than one of these
flags at once. (Likely one would implement this as a bitmap.) However,
the database configuration does not allow more than one flag
to be set at a time.


\subsection{PixelNameTranslation}

This class generates the translation between the names used in the 
naming document and the hardware addresses. This includes both the
FEC and the FED.

The data format used for the name translation is given by:
{\tiny
\begin{verbatim}
# name                          TBMchannel  FEC      mfec  mfecchannel hubaddress portadd rocid     FED     channel     roc#
FPix_BpI_D1_BLD1_PNL1_PLQ2_ROC0      A       1        8        1          31        0       0        1         12        0
FPix_BpI_D1_BLD1_PNL1_PLQ2_ROC1      A       1        8        1          31        0       1        1         12        1
FPix_BpI_D1_BLD1_PNL1_PLQ2_ROC2      A       1        8        1          31        0       2        1         12        2
FPix_BpI_D1_BLD1_PNL1_PLQ2_ROC3      A       1        8        1          31        0       3        1         12        3
FPix_BpI_D1_BLD1_PNL1_PLQ2_ROC4      A       1        8        1          31        0       4        1         12        4
FPix_BpI_D1_BLD1_PNL1_PLQ2_ROC5      A       1        8        1          31        0       5        1         12        5
FPix_BpI_D1_BLD1_PNL1_PLQ2_ROC6      A       1        8        1          31        0       6        1         12        6
FPix_BpI_D1_BLD1_PNL1_PLQ2_ROC7      A       1        8        1          31        0       7        1         12        7
FPix_BpI_D1_BLD1_PNL1_PLQ2_ROC8      A       1        8        1          31        0       8        1         12        8
FPix_BpI_D1_BLD1_PNL1_PLQ2_ROC9      A       1        8        1          31        0       9        1         12        9
\end{verbatim}
}
The name translation allows us to map the ROC name to the hardware addresses
used in the configuration.

\subsection{PixelFECConfig}

This class specifies the location of the pixel FECs. This
basically gives the VME base address to use. An arbitrary FEC
number is used here. {\it is this specified in the naming
convention document?} Also we refer to the crate number here. 
There will be one PixelFECSupervisor per crate. How is this
number identified? The PixelFECSupervisor is initialized
with a crate number and will control the FEC cards that are 
in the crate.

The file format that we have to store this information looks like
\begin{verbatim}
#FEC number     crate     vme base address
1               1         0x80000000
\end{verbatim}
Each FEC is identified by a number. The FEC is identified by the 
crate and base address.


\subsection{PixelTKFECConfig}

This class specifies the location of the tracker FECs in
the system. This
specifies the VME slot and crate used for each TKFEC board. An arbitrary TKFEC
ID string is used here. {\it is this specified in the naming
convention document?}
There will be one PixelTKFECSupervisor per crate. 
The PixelTKFECSupervisor is initialized
with a crate number and will control the TKFEC cards that are 
in the crate.

The file format that we have to store this information looks like
\begin{verbatim}
#TKFEC ID     crate     VME/PCI    slot/address
tkfec1          1                     0x1c
\end{verbatim}
Each TKFEC is identified by an ID string. (Currently this string
is arbitrary but we should use an agreed upon convention.)

Optionally, to use a PCI TKFEC, the string ``PCI" may be added between
the crate number and slot number.  ``VME" may also be specified.  If this
parameter is not specified, it defaults to VME.


\subsection{PixelFEDConfig}

This specifies how the FEDs are configured. This includes the FED
number and the VME base address. The FED number is the same as
the FED id in the Slink data. Each PixelFEDSupervisor is initialized
with a crate number corresponding to the crate it controls.

The file format that we have to store this information looks like
\begin{verbatim}
#FED number     crate     vme base address
1               1         0x1c000000
\end{verbatim}
Each FED is identified by a number, this is the same number as 
the FED id in the raw data. The FED is identified by the 
crate and base address. For now the crate is identified by an
arbitrary number. {\it Should this be changed to the 
actual name of the crate?}




\subsection{PixelCalibConfiguration}

This class was formerly known as PixelCalib, but was renamed after
it was moved to the CMSSW repository in order to make it more
consistent with offline conventions.
This class incorporates information about how a calibration
is executed. In particular it handles calibrations where 
groups of pixels have charge injects. It specifies how we loop over pixels
and pulse the detector in a calibration. The class is also
used in the offline in order to analyze the calibration data,
so that we know what event had what charge injected and what
pixel were expected to be hit. 

Below is an example of this file:
\begin{verbatim}
Mode: ThresholdCalDelay
Rows:
10 | 20 
Cols:
10 | 20
VcalHigh
Scan: VcThr 0 255 8
Scan: CalDel 0 255 8
Set: Vcal 50
Repeat: 10
Rocs:
FPix_BmI_D1_BLD1_PNL1_PLQ1_ROC0
FPix_BmI_D1_BLD1_PNL1_PLQ1_ROC1
FPix_BmI_D1_BLD1_PNL1_PLQ2_ROC0
FPix_BmI_D1_BLD1_PNL1_PLQ2_ROC1
.
.
.
\end{verbatim}

The {\tt Scan} statement allows you to specify that
you want to scan the settings of a dac parameter in a 
range, above starting at 0 and incrementing in steps of
8 until it exceeds 255. In addition to this you can
specify a non-uniform set of scan points using the
following format
\begin{verbatim}
ScanValues: Vcal 10 20 30 35 40 42 44 46 48 50 
                 52 54 56 58 60 65 70 80 90 100 -1
\end{verbatim}
The {\tt -1} 

Arbitrary parameters may be specified just before the ``Rows:" line.
For example,
\begin{verbatim}
Mode:  AOHBias
Parameters:
TargetBMin     412
TargetBMax     612
printFEDRawData             no
printFEDOffsetAdjustments   no
printAOHBiasAdjustments     no
Rows:
.
.
.
\end{verbatim}
These parameters are accessible in the calibration code.  Each calibration
may look for particular parameters to control its operation.  Parameters
not defined for a particular calibration are ignored.

In particular the parameter ``ScanMode'' can be defined. It can take
the three values of ``maskAllPixel'', ``useAllPixel'', and ``default''.
In the default mode the trim and mask bits specified in the configuration 
is used during the scan. Charge is injected according to the pattern
and the pixels that are enabled in the configuration is used during
the calibration. In the maskAllPixels mode all pixels are
disabled before the first event. Then the pixels are enabled
corresponding to the pixel mask and the pattern that has charged
injected. I.e., only pixels that have charge injection and that
are not disabled in the configuration will be enabled. The pixels
use the trim bits from the configuration. In the useAllPixels mode
all the pixels on the current pattern is enabled independently of
what the mask bit is in the configuration if ScanMode is not 
defined the mode useallPixels will be used.

DACs to scan are selected with lines of the form.
\begin{verbatim}
Scan: [DAC name] [min scan value] [max scan value] [scan step size] [mix]
\end{verbatim}
The last parameter is optional.  If nothing is given here, then all ROCs will be set to the same DAC value at the same time.  If the word \verb|mix| is placed here (at the end of the line), then the ROCs on a particular channel will have not have the same DAC value at the same time.  Instead, the DAC values on different ROCs will be spread out to cover the entire range.  This is useful when scanning \verb|Vsf| or any other setting that affects the power drawn by the chip, as it prevents the ROCs from all drawing high power at the same time.

The list of ROCs may be specified completely, or it may be auto-generated.  
Auto-generation requires knowing which modules are in the configuration, and 
which ROCs are on those modules.  Currently, this information is not 
available in offline (i.e. CMSSW) code, so offline code must have the 
ROC list specified completely.  To do this, the format is:
\begin{verbatim}
Rocs:
FPix_BpI_D1_BLD1_PNL1_PLQ2_ROC0
FPix_BpI_D1_BLD1_PNL1_PLQ2_ROC1
.
.
.
\end{verbatim}
In online running, where this information is available, it is preferable 
to use auto-generation.  To auto-generate the list, the first line should 
be ``\verb|ToCalibrate:|" instead of ``\verb|Rocs:|".  (The user may 
specify a complete ROC list with ``\verb|ToCalibrate:|".  Compared to 
using ``\verb|Rocs:|", this has the advantage of not adding ROCs which 
are not found in the configuration.  So it is always recommended to 
use ``\verb|ToCalibrate:|" if possible.)

Probably the most common thing is to just add all ROCs and modules in 
the configuration. This simply requires the word \verb|all|:
\begin{verbatim}
ToCalibrate:
all
\end{verbatim}

To specify, say, just 2 ROCs, use:
\begin{verbatim}
ToCalibrate:
FPix_BpI_D1_BLD1_PNL1_PLQ2_ROC0
FPix_BpI_D1_BLD1_PNL1_PLQ2_ROC1
\end{verbatim}

To specify, say, all the ROCs on just two modules, use:
\begin{verbatim}
FPix_BpI_D1_BLD1_PNL1
FPix_BpI_D1_BLD1_PNL2
\end{verbatim}
(The ROCs on each module are obtained from the name translation.)

You can mix and match ROCs and modules:
\begin{verbatim}
FPix_BpI_D1_BLD1_PNL1_PLQ2_ROC0
FPix_BpI_D1_BLD1_PNL2
\end{verbatim}

You can also exclude particular ROCs and modules by adding them with a 
minus sign in front (separated by whitespace):
\begin{verbatim}
ToCalibrate:
all
- FPix_BpI_D1_BLD1_PNL1_PLQ2_ROC0
- FPix_BpI_D1_BLD1_PNL2
\end{verbatim}
This fills the ROC list with all ROCs in the configuration, except 
for \verb|FPix_BpI_D1_BLD1_PNL1_PLQ2_ROC0| and for all the ROCs 
in \verb|FPix_BpI_D1_BLD1_PNL2|.

For another example:
\begin{verbatim}
ToCalibrate:
FPix_BpI_D1_BLD1_PNL1
- FPix_BpI_D1_BLD1_PNL1_PLQ2_ROC0
\end{verbatim}
This adds all ROCs on \verb|FPix_BpI_D1_BLD1_PNL1| except 
for \verb|FPix_BpI_D1_BLD1_PNL1_PLQ2_ROC0|.

You may optionally add a ``\verb|+|" in front of things that 
you're adding. (If no ``\verb|+|" or ``\verb|-|" is seen, ``\verb|+|" is 
assumed.)
\begin{verbatim}
ToCalibrate:
+ all
- FPix_BpI_D1_BLD1_PNL1
+ FPix_BpI_D1_BLD1_PNL1_PLQ2_ROC0
\end{verbatim}
This adds all ROCs on all modules, except 
for \verb|FPix_BpI_D1_BLD1_PNL1|, on which 
only \verb|FPix_BpI_D1_BLD1_PNL1_PLQ2_ROC0| is added.

Note that the order matters. The following:
\begin{verbatim}
ToCalibrate:
+ all
+ FPix_BpI_D1_BLD1_PNL1_PLQ2_ROC0
- FPix_BpI_D1_BLD1_PNL1
\end{verbatim}
would not include \verb|FPix_BpI_D1_BLD1_PNL1_PLQ2_ROC0| in the 
ROC list, because it is later removed by the 
removal of \verb|FPix_BpI_D1_BLD1_PNL1|.

(For completeness, even though it's completely useless, you 
can use ``\verb|- all|" to clear the ROC list.)

The auto-generated ROC list only adds ROCs on modules 
given in the \verb|PixelDetectorConfig|.

\subsection{PixelFedCard}

The {\tt PixelFEDCard} class contains the settings for a FED board.
The file is fairly long and I do not include an example here. 
You can find a sample file in PixelFEDInterface/test/params\_fed.dat 
{\it There is some information on the PixelFEDCard that is 
redundant with the PixelFEDConfig like the VME base address. We
should only specify this in one location.}

\subsection{PixelTBMSettings}

This class holds the settings used by the TBM. The file contains the
module name and gain settings as well as if the TBM should be
configured in 'SingleMode' or 'DualMode'. 

The format used for this information is given by
\begin{verbatim}
FPix_BpI_D1_BLD1_PNL1_PLQ2_ROC1
AnalogInputBias: 160
AnalogOutputBias: 110
AnalogOutputGain: 207
Mode: SingleMode
\end{verbatim}
{\it Note that the name here should be a module name and not contain
the plaquette and ROC number. We now have a 'PixelModuleName' class
that we should use here.}


\subsection{PixelPortcardMap}

This class lists the portcards used in the configuration
and which fed channels are on which AOH.

The format of this information is illustrated by the file:
\begin{verbatim}
# PortcardName       Module                AOH channel
FPix_BpI_D1_PRT2    FPix_BpI_D1_PNL4 A     1
FPix_BpI_D1_PRT2    FPix_BpI_D1_PNL5 A     2
\end{verbatim}
For each portcard the modules are listed and the corresponding
AOH channel is listed.  The TBM channel, ``\verb|A|" or ``\verb|B|", is
specified after the module name.  (If no TBM channel is specified, it
defaults to ``\verb|A|".)

The AOH channel numbering begins from 1.  On each forward pixel port card, there is one AOH with 6 channels, so the numbering goes from 1 to 6.  On barrel supply boards, there are 4 AOHs, each with 6 channels.  The numbering goes from 1 to 24.  The first AOH contains channels 1-6, the second contains channels 7-12, etc.


\subsection{PixelPortCardConfig}

This class holds the settings that are used on
each portcard.

The format of the portcard configuration file is
\begin{verbatim}
Name: FPix_BpI_D1_PRT1
TKFECID: tkfec1
ringAddress: 0x8
ccuAddress: 0x7d
channelAddress: 0x10
i2cSpeed: 0x64
Delay25_GCR: 0x0
Delay25_SCL: 0x60
Delay25_TRG: 0x68
Delay25_SDA: 0x5c
Delay25_RCL: 0x60
Delay25_RDA: 0x60
AOH_Bias1: 0x19
AOH_Bias2: 0x1f
AOH_Bias3: 0x1f
AOH_Bias4: 0x1f
AOH_Bias5: 0x1f
AOH_Bias6: 0x1f
AOH_Gain1: 0x2
AOH_Gain2: 0x0
AOH_Gain3: 0x0
AOH_Gain4: 0x0
AOH_Gain5: 0x0
AOH_Gain6: 0x0
\end{verbatim}
In the barrel, there are 4 AOH devices, denoted \verb|AOH1|, \verb|AOH2|, \verb|AOH3|, and \verb|AOH4|.  Each of these 4 devices has 6 channels, each with its own bias and gain.  For example, \verb|AOH1_Bias1|, \verb|AOH1_Gain5|, \verb|AOH2_Bias3|, \verb|AOH4_Gain6|.

In addition, values may be specified for the settings \verb|PLL_CTR1|, \verb|PLL_CTR2|, \verb|PLL_CTR3|, \verb|PLL_CTR4or5|, \verb|PLL_CTR4|, \verb|PLL_CTR5|, \verb|DOH_Ch0Bias_CLK|, \verb|DOH_Dummy|, \verb|DOH_Ch1Bias_Data|, and \verb|DOH_Gain_SEU|.  The settings \verb|PLL_CTR4or5|, \verb|PLL_CTR4|, and \verb|PLL_CTR5| are a special case.  \verb|PLL_CTR4or5| is the actual address that may be written.  If bit 5 of \verb|PLL_CTR2| is set to zero, this address writes to \verb|PLL_CTR4|; if that bit is set to one, this address writes to \verb|PLL_CTR5|.  When \verb|PLL_CTR4| or \verb|PLL_CTR5| is used in the configuration file, this bit will be automatically set correctly without needing to manually specify a setting for \verb|PLL_CTR2|.  It is recommended to use this feature, instead of using \verb|PLL_CTR4or5|.

Prior to loading the settings in the file, an initialization sequence is sent to the port card:
\begin{verbatim}
PLL_CTR1:    0x8
PLL_CTR1:    0x0
PLL_CTR2:    0x20
PLL_CTR4or5: 0x15
DOH_Dummy:   0x0
\end{verbatim}
Then the device settings in the file (all lines after \verb|i2cSpeed: 0x64|) are loaded in the order given.

\subsection{PixelDelay25Calibration}

This class specifies the parameters used in the delay 25 scan
for the delay settings of return data and send data. An
example of the configuration files used for this calibration
is given below

\begin{verbatim}
Mode:
Delay25
Portcards:
FPix_BmI_D1_PRT2
AllModules:
0
OrigSDa:
64
OrigRDa:
64
Range:
64
GridSize:
8
Tests:
10
StableRange:
6
StableShape:
2
\end{verbatim}


\subsection{PixelGlobalDelay25}

The PixelGlobalDelay25 is a class that contains one global
delay setting that delays the signal in delay25 chip for the
clock and data. In the FED it adds the same delay such that
the digitization works independently of the delay setting
in the global delay.

The format of this file is very simple, it is just one single
number in hex
\begin{verbatim}
0x10
\end{verbatim}
The delay specified here is in units of 0.499 ns. This 
corresponds to the steps of the delay in the delay 25 chip.

