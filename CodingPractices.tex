% Coding practices

\subsection{Makefile}

The pixel online software is built with a
Makefile located in each (sub)package. The
Makefile builds on the xdaq tools. This
file should ideally be as short as possible
and use the functionallity in the xdaq package.
You invoke the Makefile in each package by 
doing a 'make'. There is also a {\tt clean} target.

In addition to the package level makefiles 
this is a toplevel makefile in the pixel
directory. This makefile allow you to build 
all the xdaq packages using {\tt make Set=pixel}.
You can also clean all packages using 
{\tt make Set=pixel clean}.

\subsection{Include files}

Include statements should include the file path 
starting from the project. For example you should do

\begin{verbatim}
#include "PixelCalibrations/include/PixelAOHBiasCalibration.h"
#include "CalibFormats/SiPixelObjects/interface/PixelCalibConfiguration.h"
\end{verbatim}

\subsection{CVS tags}

We create 'official' tags of the form 'POS\_X\_Y\_Z',
e.g. 'POS\_2\_4\_5'. For every tag created, starting
with POS\_2\_5\_0, an entry should be made in the
file pixel/README that describes briefly the new
features in the tag.

\subsection{Building RPMs}

The building of RPMs should be straightforward. The
following steps are required.
\begin{itemize}
\item Prepare the code, update the README file, and VERSION file and
      commit and tag the code. The VERSION file contains the version
      of the RPM to build.
\item Invoke 'make Set=pixel rpm' to build RPMs for all pixel online software 
      packages.
\item In the {\tt PixelUtilities} directory, invoke 'buildExternalRPMs.sh'
      to build RPMs for DiagSystem, TTCSoftware, and FecSoftwareV3\_0.
      At the end of building the external RPMs this script copies
      all RPMs to \$BUILD\_HOME/RPM\_X.Y.Z-V, where X, Y, Z, and V are
      the major version, minor version, patch, and build version, 
      respectively.
\end{itemize}
The set of RPMs built can be tested (in the online environment) for
consistency by invoking the command 'rpm --test -Uvh *.rpm' in the directly
where all the RPMs are located.

Old RPMs can be cleaned out of the build area using the command 'make Set=pixel cleanrpm'.

\clearpage

