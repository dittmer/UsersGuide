% Configuration Database Interface

Since early summer 2006, we have used an interface for accessing
configuration data in the online software framework. 
The access was originally file based, and has now (starting in 2009) been
supplemented with infrastructure to access an Oracle database. The 
interface used is rather generic and the purpose of this document
is to write down the interface so that we can have a 
clear separation between the database code and the applications 
that use data from the configuration database.

First in Sect.~\ref{sect:cppinterface} we describe the C++ API.
The interface is defined in the class {\tt PixelConfigInterface}.
This interface is fully implemented in the {\tt PixelConfigFile}
implementation based on files.

In Sect.~\ref{sect:cmdline} a simple command line tool is described.
This tool is implemented through the C++ interface and should work
both for the file based and the final data base implementation.

Examples below use a few of the configuration data classes that are used. 
% In Sect.~\ref{sect:configclasses} the full
% set of classes that are used to configure and control the pixel
% online software are discussed.

\subsection{C++ Configuration Data Access API}
\label{sect:cppinterface} 

The class {\tt PixelConfigInterface} in the package 
{\tt PixelConfigDBInterface} defines the interface for
access of configuration data. This interface is intended
to provide type safe access methods for retrieving and
storing configuration data.

\subsubsection{Retrieving data from the database}

The primary method for retrieving the data is
\begin{verbatim}
  template <class T>
  static void get(T* &pixelObject,
		  std::string path,
		  pos::PixelConfigKey key)
\end{verbatim}

The {\tt PixelConfigKey} is just an integer that holds the top
level configuration key to be used. 
This interface returns a pointer to the data. The caller is assumed
to take ownership of the data and delete it. 
The path is a 'secondary key'. It would allow us to store more than
one object of the same type in a given configuration. (In the file
based implementation this label is used to build the path to were the
file is stored.)
Below are a few example of using this interface.
\begin{verbatim}

  PixelConfigKey theGlobalKey(5);

  PixelNameTranslation *theNameTranslation=0;
  PixelConfigInterface::get(theNameTranslation, "nametranslation/", 
                     theGlobalKey);

  PixelDetectorConfig *theDetectorConfiguration=0;
  PixelConfigInterface::get(theDetectorConfiguration, "detconfig/", 
                     theGlobalKey);

  PixelFECConfig *theFECConfiguration=0;
  PixelConfigInterface::get(theFECConfiguration, "fecconfig/", 
                     theGlobalKey);

\end{verbatim}

These examples show how you extract objects for which there is only one 
instance of for the
whole detector configuration. You pass in as the first argument a pointer. The 
pointer will return 0 if the object was not successfully retrieved.\footnote{We have now tried to improve on this error-handling scheme by throwing a {\tt std::exception} in case there is a failure to retrieve the configuration information. When we attempt to retrieve configuration data in the POS software, we both test for a null pointer and handle any exceptions thrown.} The
second argument is the label for the object. This is essentially a key that
is used to look up the data. The interface is type safe, i.e., if you specify 
a path to an object of the wrong type you will get back a null pointer. The
third argument is the global configuration key. This basically specifies 
the versions of all objects used in a given configuration. This key is
implemented as an integer.

Besides objects like the name translation and detector configuration listed
above, there are objects such as trim bits, mask bits, and dac values 
that we need
to access on a finer granularity than for the whole detector. To do this
we use slightly modified arguments

\begin{verbatim}
  PixelDACSettings *tempDACs=0;
  PixelConfigInterface::get(tempDACs,''pixel/dac/FPix_BpI_D1_BLD1_PNL1'', 
                     theGlobalKey);
\end{verbatim}

where we have added to the path the module name for which we want to 
extract the
dac settings.

As a given application, for example the PixelFECSupervisor, will need to access
dac settings for many modules, and it is more efficient to extract
the data 'in bulk' from the database, we have also added an interface
that allows extraction of multiple objects at the time

\begin{verbatim}
  std::map<std::string, PixelDACSettings*> dacs;
  dacs["pixel/dac/FPix_BpI_D1_BLD1_PNL1"]=0;
  dacs["pixel/dac/FPix_BpI_D1_BLD1_PNL2"]=0;
  PixelConfigInterface::get(dacs, theGlobalKey);
\end{verbatim}

The method will add pointers to the modules; if a module is not
in the configuration a null pointer is returned.

In addition to the access method that uses the configuration key,
you can also retrieve data based on the specific version of the
object
\begin{verbatim}
  template <class T>
  static void get(T* &pixelObject,
                  std::string path,
		  unsigned int version)
\end{verbatim}
There is also a version that retrieves all the objects
\begin{verbatim}
  template <class T>
  static void get(std::map<std::string, T*> &pixelObjects,
		  unsigned int version)
\end{verbatim}
In general the access based on the configuration key should be used
in the xdaq applications.


\subsubsection{Storing data in the database}

To store data in the configuration database use the following 
method

\begin{verbatim}
  template <class T>
  int PixelConfigInterface::put(const T* ,std::string path);
\end{verbatim}

This method will install the data in T* using the specified path.
The new version number is returned by the method.
If you instead have a set of configuration data that needs to be
installed you use the interface

\begin{verbatim}
  template <class T>
  int PixelConfigInterface::put(std::vector<T*>,
                                std::string path);
\end{verbatim}

Where a vector of pairs is passed, the second argument in the
pair is the sub path used for the object.

\subsubsection{Configuration keys}

A configuration key consists of a set of objects and their versions.
The simplest form is just to specify a list of the paths and the 
versions to use in the key
\begin{verbatim}
  static unsigned int makeKey(std::vector<
                              std::pair<std::string, unsigned int> > versions)
\end{verbatim}
The method returns the new configuration key.


\subsubsection{Alias manipulation}

For configurations aliases you can retrieve the list of defined 
aliases using
\begin{verbatim}
  static std::vector<std::pair<std::string, unsigned int> > getAliases()
\end{verbatim}
The string is the name of the alias and the unsigned int is the 
corresponding key. {\it I think it should actually be a PixelConfigKey.}

A very similar method is
\begin{verbatim}
static std::map<std::string, unsigned int> getAliases_map()
\end{verbatim}
That returns a map between the alias name and the corresponding key.

The method
\begin{verbatim}
static void addAlias(std::string alias, unsigned int key)
\end{verbatim}
inserts a new alias. Note that if the alias already exists it will
just be updated to point to the new key. 

The method below is a lower level method. This method
allows you to define an alias to a version that has key
already created and link it to version aliases. This method
should eventually be removed as it is probably to error prone...
\begin{verbatim}
  static void addAlias(std::string alias, 
                       unsigned int key,
		       std::vector<std::pair<std::string, std::string> > versionaliases)
\end{verbatim}

To create an alias for a configuration object use the method.
\begin{verbatim}
    static void addVersionAlias(std::string path, 
                                unsigned int version, 
                                std::string alias)
\end{verbatim}

To add an alias for a configuration object, as supposed to a
configuration key, use

\begin{verbatim}
  static void addVersionAlias(std::string path,
                              unsigned int version, 
                               std::string alias)
\end{verbatim}


To get the actual version that an alias points to use the method
\begin{verbatim}
  static unsigned int getVersion(std::string path,
                                 std::string alias)
\end{verbatim}

\subsubsection{Support of polymorphism}

The interface should support polymorphism in the sense described
below. For example, for trim bits it may be convenient to have ways of
storing the trim bits in different ways. We will need to be able 
to store trim bits so that we can set them independently for
each pixel. But there are other cases where we might want to set
all trim bits the same on a whole ROC, or the same in each 
double column.

Consider that we have a base class, {\tt PixelTrimBase} and that there
are the concrete implementations {\tt PixelTrimAll}, {\tt PixelTrimROC},
and {\tt PixelTrimDCol} which implements the per pixel, per roc, and
per double column respectively. The interface should support
operations like

\begin{verbatim}
  PixelTrimBase *tempTrims=0;
  PixelConfigInterface::get(tempTrims,"pixel/trim/FPix_BpI_D1_BLD1_PNL1", 
                     theGlobalKey);
\end{verbatim}

where after {\tt tempTrims} points to the data type stored in the 
database for the configuration.

Similarly you should be able to store data

\begin{verbatim}
  PixelTrimBase *tempTrims=new PixelTrimROC; //probably would not compile as
                                             //we don't have this constructor.
  unsigned int ver=PixelConfigInterface::put(tempTrims,"pixel/trim/FPix_BpI_D1_BLD1_PNL1");
\end{verbatim}



\subsection{Command line interface to configuration data}
\label{sect:cmdline}

Based on the interface described above in the C++ interface a simple
command line tool has been written to allow manipulation of configuration
data. The functionality of this interface is described below.

\subsubsection{Inserting new data}

Here we will start be discussing how you insert a
{\tt PixelDetectorConfig} object. There is
only one such object in the configuration. Assuming that we
have this in a file named {\tt detconfig.dat}. We now want to
insert this into the configuration database. This is done with

\begin{verbatim}
PixelConfigDBCmd.exe --insertData detconfig detectconfig.dat
\end{verbatim}
This would install the content of the file {\tt detconfig.dat} 
under the path {\tt detconfig} by creating a new version.
This new version is returned by the command so that it is known
where it was installed. The implementation of this tool
reads the file to create a {\tt PixelDetectorConfig} object and
then uses the C++ interface to store the data.

As a variation of this you might want to install, e.g., a new set of 
DAC settings for the ROCs. As we insist that no data can be 
changed in the configuration database after it has been loaded
this implies that the DAC settings for all ROCs needs to be
loaded at once. To insert that prepare the files that contains the
ROC dac settings. Then create the file daclist.txt that list all
the file that you want to install. To install them into the
database us

\begin{verbatim}
PixelConfigDBCmd.exe --insertDataSet dac daclist.txt
\end{verbatim}

Again a new version has been created under the path {\tt dac} and
all the dac settings have been uploaded. The new version is then
printed out.

Note that these interfaces are designed to not allow
you to change any existing data. (We might want to consider allowing
adding comments to existing versions of configuration data.)

\subsubsection{Retrieving data}

From the command line you can retrieve data using
\begin{verbatim}
PixelConfigDBCmd.exe --getVersion nametranslation/ 0
\end{verbatim}
This will retrieve the data from version 0 of the nametranslation 
and write it out as a file.

\subsubsection{Creating new configurations}

So far we have discussed how to add new data to the configuration
database. The next step is to combine versions of several 
data types into a configuration. This is done with a command 
like

\begin{verbatim}
PixelConfigDBCmd.exe --insertConfigAlias Physics dac 1 detconfig 0 nametranslation 0
\end{verbatim}

where the versions of the different objects on the paths are 
listed. This will create a new configuration key. This key will
be returned by the command. 

Very often we have an existing configuration, {\tt oldKey}, that
we want to 'update'. Note that update actually means that we will create
a new configuration. For example you can do
\begin{verbatim}
PixelConfigDBCommand --updateConfigAlias 5  tbm 6
\end{verbatim}
{\it not yet implemented.} So in the example above a copy of 
configuration key number 5 would
be made in which the TBM settings version 6 was added.

If one wants to remove an existing object from a configuration
we could imagine doing something like
\begin{verbatim}
PixelConfigDBCommand --updateConfigAlias 5  calib -1
\end{verbatim}
 
In both these cases the new configuration key that was created is
returned. Again this interface guarantees that no existing 
data has been changed. Only new data has been added.


\subsubsection{Alias manipulation}


The simplest use case is that we can create an alias that points
to a specific key. This could e.g. be to point the 'Physics' 
alias to configuration key 5
\begin{verbatim}
PixelConfigDBCommand --insertConfigAlias Physics 5
\end{verbatim}

But we need to expand on this to make it more manageable
to handle the configurations.

In addition to have aliases for the toplevel configurations
we can define aliases for the configuration data versions.
For example you can define the 'Physics' alias for the TBM settings
version 4 using
\begin{verbatim}
PixelConfigDBCommand --insertVersionAlias tbm 4 Default
\end{verbatim}
Very similar to the {\tt --createconfig} from above you
can also do
\begin{verbatim}
PixelConfigDBCommand --insertConfigAlias Physics dac 5 detconfig 3 tbm Physics
\end{verbatim}
This will create a new toplevel configuration key and the 
alias 'Physics' will point to this key. Say that you then also do
\begin{verbatim}
PixelConfigDBCommand --insertConfigAlias PhysicsLowLumi dac 4 detconfig 3 tbm Physics
\end{verbatim}
We have now  created two aliases to two toplevel keys. Now, if you 
do
\begin{verbatim}
PixelConfigDBCommand --insertVersionAlias tbm 5 Physics
\end{verbatim}
this should automatically create two new toplevel configurations
for 'Physics' and 'PhysicsLowLumi'.

\subsubsection{Use cases}

Below I will give some examples for how we could be using the
capabilities discussed in the previous two sections. To simplify
this a little bit I will work with a configuration that only
has 4 different types of objects (detconfig, tbm, dac, and masks).

The first thing we need to do is to load the configuration data.
Assume that we do this on a completely empty database.

\begin{verbatim}
PixelConfigDBCmd.exe --insertData fecconfig fecconfig.dat
PixelConfigDBCmd.exe --insertData fedconfig fedconfig.dat
PixelConfigDBCmd.exe --insertData tkfecconfig tkfecconfig.dat
PixelConfigDBCmd.exe --insertData portcardmap portcardmap.dat
PixelConfigDBCmd.exe --insertDataSet fedcard fedcardlist.txt
PixelConfigDBCmd.exe --insertDataSet portcard portcardlist.txt
PixelConfigDBCmd.exe --insertDataSet tbm tbmlist.txt
PixelConfigDBCmd.exe --insertDataSet dac daclist.txt
\end{verbatim}
So now we have version 0 for each of these objects. Now I can
create aliases for the different objects.

\begin{verbatim}
PixelConfigDBCommand --insertVersionAlias detconfig 0 Physics
PixelConfigDBCommand --insertVersionAlias tbm 0 Default
PixelConfigDBCommand --insertVersionAlias dac 0 Default
PixelConfigDBCommand --insertVersionAlias mask 0 Default
\end{verbatim}

\begin{verbatim}
PixelConfigDBCommand --insertConfigAlias Physics dac Default detconfig Physics
tbm Default mask Default
\end{verbatim}

This command will create the first configuration key, number 0. Now
say that a new set of dac settings is created and loaded:
\begin{verbatim}
PixelConfigDBCommand --insertData dac daclist.txt 
\end{verbatim}
This will be version 1 of the dac settings. Next if you make this
alias the 'Default':
\begin{verbatim}
PixelConfigDBCommand --insertVersionAlias dac 1 Default
\end{verbatim}
The code will automatically update all toplevel aliases that are
using this the 'Default' dac settings. In this example it
means that the top level 'Physics' alias will point to key 1.




\subsection{Managing Configurations}

We need higher level tools to manage configurations. The way we 
currently have the configurations implemented there are about 
15 different objects that are needed to build a configuration.
Two examples of such configurations are:

\begin{verbatim}
key 0
detconfig   2
nametranslation   0
fedconfig   0
fecconfig   0
fedcard   0
dac   2
mask   2
trim   2
calib   0
tbm   0
portcard   0
portcardmap   0
ttcciconfig 0
tkfecconfig 0

key 1
detconfig   2
nametranslation   0
fedconfig   0
fecconfig   0
fedcard   0
dac   2
mask   2
trim   2
calib   8
tbm   0
portcard   0
portcardmap   0
ttcciconfig 0
tkfecconfig 0
\end{verbatim}

For a moment I will not discuss the 'aliases', I will focus just
on the data organization and the tools needed to manipulate the
data. The different data types used in the configurations are 
discussed in Sect.~\ref{sect:configobjects}


\subsection{Configuration DB Implementation}

{\it to be added. Stay tuned.}


