% Outstanding tasks and improvements

I will divide these task into three different categories.
The first is algorithm/calibration development. The second
is 'operations', the third is code improvements.

\subsection{Calibration development}
\begin{itemize}
\item Implement algorithm for time walk. Some of the tools are
      available. Need to understand how to analyze the data
      and set parameters.
\item Adjust AOH gain setting. Need to discuss a strategy for this.
\item Need to fix the delay25 calibration; it wraps around.
\item Linearity adjustment. Steve started to look at Vsf and Vhlddel.
\item Adjust the ROC analog signal offset and gain. Want to keep the
      lowest level well above UB, and the maximum near 255 (or 1024).
\item In the address level determination if the highest level hit 1023 it 
      will not be detected. Should allow this.
\end{itemize}

\subsection{Operation improvements}

\begin{itemize}
\item Migrate calibration algorithms to use root objects, like
      histograms. This will allow us to write out root files
      with the calibration results as well as monitoring
      the progress of calibrations.
\item Tools to view the calibration results should be 
      migrated to look at root files.
\item Deploy database. 
\item Use filebased interface and aliases to nor write over 
      old files. This is basically done. But should modify the
      writing of files so that all files in the configuration
      is written out again. Otherwise it will be to hard to used.
      This is not very complicated, but not sure what is the 
      best strategy.
\item Deploy the error logger to make sure that we send relevant
      and not to verbose messages.
\item Have to run with the FEDSpySupervisor.
\item Run with the power-on sequence. This is work in progress
      now and a first try will take place during the week of 
      Jan. 28, 2008.
\item We should properly initialize the CCU. 
\item We should try out the schemes for reconfiguring a CCU ring
      to drop a CCU.
\item Is it possible that we can program ROCs on one panel, or blade,
      such that we can burn the fuse on the adapter board?
\item We have to try out the 'popcon' feature for calibrations.
\item Embed the 'private' words in the FED data. Have code example
      now from Will. Need to modify the rawToDigi code to unpack
      this information.
\end{itemize}

\subsection{Code improvements}

\begin{itemize}
\item Move DCU readout workloop to configure transition.
\item Calibration algorithms should run on separate threads.
      This is mostly implemented now. Some calibrations 
      still need to migrate to executing the calibration
      in 'steps', and not just as on call. 
\item should use dynamic cast instead of static cast.
\item Further cleanup in calibrations. A number of the
      calibrations contains a large amount of duplication.
\item Some code like the FEDInterface and FEDCard is too verbose.
      Need to review printouts.
\item There are some exceptions we need to catch. For example
      if we try to talk to a non-existing FED base address we
      should catch the exception and not just crash.
\item Should the supervisor applications be self updating like
      the trigger supervisor applications? This would then allow
      to automatically update progress status during a calibration
      and handle when a calibration is complete.
\item The ROC status should be respected. This would, e.g., allow
      us to not generate a failed calibraton when one of the
      ROCs can't generate hits.
\item More consistency checking needed; in particular for the AOH
      initialization. Should only initialize portcards that are
      used.
\item When calibration complete the event processing the FED Supervisor
      displays the page that indicates that the calibration is done
      while the FED is still doing the endCalibration processing.
\end{itemize}



\clearpage

\bibliographystyle{unsrt}
\bibliography{UsersGuide}



\end{document}



