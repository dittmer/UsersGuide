% Configuration

The configuration of the detector starts with a global key, which
encodes the set of configuration data that should be used. The
PixelSupervisor then coordinates the actions of the other
supervisors. Configuration data must be fetched from files or the
database, and then the data must be loaded into the detector
hardware. This happens in several steps that must happen in a
particular order. The logic governing this process is described below.

\subsection{Configuration Sequence}
The configuration sequence is listed below.
\begin{itemize}
\item PixelSupervisor::Configure
\begin{itemize}
\item Translate the configuration alias into the Global Key
\item Send preConfigure command and Global Key to the PixelFECSupervisors
\end{itemize}
\item PixelSupervisor::transitionHaltedToConfiguring
\begin{itemize}
\item Fetch other miscellaneous configuration data from the database
\end{itemize}
\item PixelSupervisor::stateConfiguring
\begin{itemize}
\item Configuration of PixelDCSFSMInterface
\item (wait for completion)
\item Parallel configuration of PixelTKFECSupervisors
\item (wait for completion)
\item Parallel configuration of PixelFECSupervisors
\item Parallel configuration of PixelFEDSupervisors
\item (wait for completion)
\item Configuration of TTCciControl
\item Configuration of PixelLTCSupervisor
\item Configuration of PixelDCStoTrkFECDpInterface
\end{itemize}
\end{itemize}
The list covers the steps visible to the PixelSupervisor. Each
individual supervisor also has several internal configuration steps
that are not described here. Also note that the only configuration
information passed from the PixelSupervisor to the other supervisors
is the global key. Each supervisor is responsible for fetching its own
configuration data from the database, based on this global key.

As shown in the list, for several of the configuration steps we must
wait for them to complete before moving on to the next step. This is
implemented using objects of type {\tt std::map<instance, state>}
maintained by the PixelSupervisor. These objects track the state of
the underlying supervisors\footnote{Note that these objects are useful for more than just configuration, and are always kept updated with the state of each Supervisor}, and are called:
\begin{verbatim}
statePixelFECSupervisors_
statePixelFEDSupervisors_
statePixelTKFECSupervisors_
statePixelDCSFSMInterface_
\end{verbatim}

These are initialized in {\tt PixelSupervisor::Initialize} by actively
asking each supervisor with a SOAP command ``FSMStateRequest''.

Note that the first step in the configuration after loading the global key is to send it to the FECSupervisors.
Allow the FECSupervisors cannot begin programming the hardware until later in the configuration sequence, they can
then begin to immediately fetch configuration data from the database. This step, called preConfiguration, is done
outside of the finite state machine structure (the PixelFECSupervisors remain in the Halted state even as they do the
preConfiguration). This step saves considerable time in the configuration process (roughly 25 seconds when configuring
from the database).

In the {\tt PixelSupervisor::stateConfiguring} method, 
the PixelSupervisor loops over all
the PixelTKFECSupervisors, checks the FSM state of each as maintained in
{\tt statePixelTKFECSupervisors\_}, and if it finds ``Halted'', it tries to
``Configure'' it and updates the map to read FSM state ``Configuring''. 
If any of these PixelTKFECSupervisors are not in the ``Configured'' state,
the ::stateConfiguring function exits. The PixelSupervisor then does nothing until it receives a 
``FSMStateNotification'' SOAP message from PixelTKFECSupervisor with the message ``Configured''.
This FSMStateNotification signals the
PixelSupervisor to make a transition from ``Configuring'' to ``Configuring'',
thus triggering ::stateConfiguring to run again.
In this way the PixelTKFECSupervisors configure in parallel, and can take as long as they like to finish configuration.

Once the TKFECSupervisors are done, we can configure the pixel FECs and FEDs in
parallel. The procedure is similar: if their FSM state is ``Halted'', we give them
the SOAP message to Configure and set their local map state to
``Configuring''. And now we can configure the FEDs too in a similar,
parallel, manner. (Note that we do not return out of the function directly, but rather if we
detect one of the supervisors to not be in ``Configured'', we
set a bool called ``proceed'' to false, and this bypasses a large chunk
of code that configures the LTC and TTC, and transitions the FSM of
PixelSupervisor to ``Configured''.) Only when all
FECs and FEDs are configured, then we proceed to configure the TTC \& LTC, and then
the PixelSupervisor is pushed into the ``Configured'' state.

\subsubsection{Possible modifications to the configuration Sequence}

In principle there is nothing to prevent the TTC and LTC from being
configured before the other Supervisors. Arguably, this would be more logical, since the trigger source
should be configured before the other devices are programmed.

\subsection{Configuration steps of the underlying supervisors}
In this section we give an outline of the steps taken by each supervisor during configuration.

\subsubsection{PixelDCSFSMInterface and PixelDCStoTrkFECDpInterface}

The PixelDCSFSMInterface loads the detector configuration. If it finds
that an entire ROG is in {\tt noInit} or {\tt noAnalogSignal}, then it
ignored that ROG when summarizing the power state of that section of
the detector. Note that initially this was true only for ``noInit'',
since a ROG marked {\tt noAnalogSignal} will still be configured and
thus needs LV to be on. This logic was changed to accomodate turning
on only a small fraction of the HV during the first beam operations.

Note that presently the relationship between detector modules (as
listed in the detconfig) and DCS ROGs is hard-coded in this class. See
Sec.~\ref{sec:dcshardcode} for more information.

\subsubsection{PixelTKFECSupervisor}

{\it Need to look through the code to verify this info!}
\begin{itemize}
\item Create a FecAccess object and issue a VME bus reset
\item Issue a resetPlxFec to reset the CCU and portcard devices
\item Load the portcard configuration data from the database and program the portcard devices (AOH, DOH, Delay25, etc)
\end{itemize}

Note that in this last step the data is programmed to the hardware in
the same order that it is provided by the class that loads the data
from the configuration files or database. That class passes file-based
data directly, in the same order as it appears in the files. For the
database, it applies a crude sorting algorithm to put PLL settings
first, Delay25 settings second, and AOH settings last. Any other
settings come between the Delay25 and AOH settings.

\subsubsection{PixelFEDSupervisor}
%a lot of things happen here...ttcrx resets, etc
%it would be good to desribe this

\subsubsection{PixelFECSupervisor}
%should note the order of roc programming, tbm resets, etc


%could add TTCciControl, LTC

\subsection{Quick reconfiguration for the fine delay scan}

A special ``Reconfiguration'' option is available for changing the
settings that are directly relevant to the global delay of the
detector with respect to the trigger and the clock.

\subsubsection{Relevant settings}

The following settings are modified during the reconfiguration process:
\begin{itemize}
\item PixelTKFECSupervisor: the SCL, TRG, and SDA registers of the Delay25 chip
\item PixelFECSupervisor: the WBC DAC setting. Note that in the current implementation, {\bf all} DAC settings are reloaded and reprogrammed during reconfiguration.
\item PixelFEDSupervisor: the TTCrx delay
\end{itemize}

\subsubsection{Implementation}

This feature is implemented outside of the normal finite state
machine. Instead, the PixelSupervisor displays a list of configuration
aliases when we are in either the Configured state (i.e.\ after
Configuration is done, but before receiving the Start command) or the
Paused state. This list of aliases is only shown if we have been
configured for Physics, and if a special environment variable is
set. Only aliases that begin with the string ``Physics'' are shown.

The user can then select one of the displayed aliases and click ``Reconfigure''. The reconfiguration process takes about 20 seconds, and when it is finished the global key shown in PixelSupervisor will update to the new value.

\subsubsection{Limitations}

The current implementation has the following limitations:

\begin{itemize}
\item The alias used for Reconfiguration must have a globaldelay25 object
defined as part of the configuration. Note that this is stricter than
the requirement for regular configuration.

\item If the Reconfiguration fails for some reason, then the PixelSupervisor will likely still update the displayed global key. The user should be careful to check the LogReader or log file for an error message.

\end{itemize}

