% Usage of configuration data in xdaq applications

This section contains a brief description of how the configuration data 
objects are used. 

First we look at the PixelSupervisor. RCMS, via the PixelFunctionManager,
will pass a string for a configuration alias to the PixelSupervisor
during the configure transition. The PixelSupervisor gets the configuration
alias and looks up the corresponding configuration key:
\begin{verbatim}
  std::string alias=parametersReceived[0].value_;
  unsigned int globalKey=PixelConfigDB::getAliases_map().find(alias)->second;
  theGlobalKey_=new PixelConfigKey(globalKey);
\end{verbatim}
{\it This code should be fixed so that it catches if the alias does
not exist. This can be an assert, as the choices for the alias are
listed from the same map and not being able to find it is an internal
error.}

Having obtained the configuration key this is what is used to extract
configuration data. For example the PixelSupervisor extracts some
objects:
\begin{verbatim}
  PixelConfigDB::get(theCalibObject_, "pixel/calib/", *theGlobalKey_);
  PixelConfigDB::get(theDetectorConfiguration_, "pixel/detconfig/", *theGlobalKey_);
\end{verbatim}
The get method returns a pointer that the PixelSupervisor is responsible
for deleting.

When the PixelSupervisor asks the other supervisors to configure it does
this by passing the configuration key, not the alias, to them. This
guarantees that the configuration used by the different supervisors 
is consistent. We only need to retain the configuration key as a record
of how the detector was configured.

Inside the PixelFECSupervisor in the configuration method we have code 
like

{\it Note that this code snippet is a bit out of date, although the idea remains the same.}
\begin{verbatim}
  PixelConfigDB::get(theNameTranslation_, "pixel/nametranslation/", *theGlobalKey_);
  assert(theNameTranslation_!=0);

  PixelConfigDB::get(theDetectorConfiguration_, "pixel/detconfig/", *theGlobalKey_);
  assert(theDetectorConfiguration_!=0);

  PixelConfigDB::get(theFECConfiguration_, "pixel/fecconfig/", *theGlobalKey_);
  assert(theFECConfiguration_!=0);
  assert(theFECConfiguration_->getNFECBoards()==1); //FIXME

  PixelConfigDB::get(theCalibObject_, "pixel/calib/", *theGlobalKey_);
  calibStateCounter_=0;

  // Loop over all modules in the Detector Configuration and instantiate FECInterfaces required in this crate.
  // Download TBM, DAC, Masks and Trim settings into hardware.
  std::vector <PixelModuleName>::iterator module_name = theDetectorConfiguration_->getModuleList().begin();
  for (;module_name!=theDetectorConfiguration_->getModuleList().end();++module_name)
 {
   diagService_->reportError("Congiguring module=" + module_name->modulename(),DIAGDEBUG);
   const PixelHdwAddress* module_hdwaddress=theNameTranslation_->getHdwAddress(*module_name);

  unsigned int fecnumber=module_hdwaddress->fecnumber();
  unsigned int feccrate=theFECConfiguration_->crateFromFECNumber(fecnumber);
  unsigned int fecVMEBaseAddress=theFECConfiguration_->VMEBaseAddressFromFECNumber(fecnumber);

  if (feccrate==crate_){
    PixelMaskBase *tempMask=0;
    PixelTrimBase *tempTrims=0;
    PixelDACSettings *tempDACs=0;
    PixelTBMSettings *tempTBMs=0;
    std::string modulePath=(module_name->modulename());

    PixelFECInterface* tempFECInterface=new PixelFECInterface(fecVMEBaseAddress, aBHandle);
    assert(tempFECInterface!=0);
    tempFECInterface->setssid(4);

    PixelConfigDB::get(tempMask, "pixel/mask/"+modulePath, *theGlobalKey_);
    assert(tempMask!=0);
    theMasks_.insert(make_pair(*module_name, tempMask));

    PixelConfigDB::get(tempTrims, "pixel/trim/"+modulePath, *theGlobalKey_);
    assert(tempTrims!=0);
    theTrims_.insert(make_pair(*module_name, tempTrims));

    PixelConfigDB::get(tempDACs, "pixel/dac/"+modulePath, *theGlobalKey_);
    assert(tempDACs!=0);
    theDACs_.insert(make_pair(*module_name, tempDACs));

    PixelConfigDB::get(tempTBMs, "pixel/tbm/"+modulePath, *theGlobalKey_);
    assert(tempTBMs!=0);
    theTBMs_.insert(make_pair(*module_name, tempTBMs));

    tempDACs->generateConfiguration(tempFECInterface, theNameTranslation_);
    tempTBMs->generateConfiguration(tempFECInterface, theNameTranslation_);
    tempTrims->generateConfiguration(tempFECInterface, theNameTranslation_, *tempMask);

    FECInterface[fecVMEBaseAddress]=tempFECInterface;
  }
 }
\end{verbatim}

Similar extractions of the configuration data is used by other
supervisors. The supervisors cache the data received. In general, configuration data
is cleared in the halt transition. However, we now hold on to most of the configuration data in the FECSupervisors, and only
clear it before the next configuration if we see that the value of the global key has changed. In this way, we can avoid reloading
identical data from the database on the next configuration.

\subsection{Global delay 25 usage}

The global delay 25 does not need to be included in a configuration. Then it
is simply ignored if the PixelConfigInterface::get call returns a 
null pointer. However, if it exists in the configuration it has the
following effects in the different applications:

In the PixelTKFECSupervisor it is checked that we are in a physics
run and have the global delay 25 settings. If this is the case then
SCL and TRG are delayed by the delay setting in the global delay 25.
{\it It is checked in the code if the calculated delay 25 setting is 
larger than 127. If this is the case then an error message is printed
and the delay 25 without the global delay is applied. This is a little
bit dangerous as it is easy to miss such a message. Should think about 
a safer way of handling this.}
For the SDA, if the calculated delay25 setting is to large, it allows
the SDA to wrap around. 

In the PixelFEDSupervisor the a similar logic is applied; if you
have the global delay 25 and you are taking a physics run then 
the TTC RX chip adds a delay. This delay is calculated using the 
method {\tt PixelGlobalDelay25::getTTCrxDelay}.

Comments by Anders: I think that we should modify this code such that
the global delay25 delay is applied when your run all calibrations
except for the Delay25 scans. This will allow simpler tests to make
sure that the delays are properly applied, e.g., by running the 
address level calibration with different global delay 25 settings.

\clearpage

