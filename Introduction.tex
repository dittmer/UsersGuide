% Introduction

This note is intended to describe the design and implementation of the
pixel online software. The pixel online software is used for controlling and calibrating the CMS pixel
detector. In particular, it performs the following functions:
\begin{itemize}
\item Configure the detector
\item Perform online calibrations
%JMT: isn't this outside the scope of POS?
\item Analyze calibration data in online farm (CMSSW)
\item Monitor the detector during data taking
%\item Control the detector environment (DCS)
\end{itemize}

In Section~\ref{sec:PixelDAQSystem} the hardware components are introduced.
In Section~\ref{sect:overview} the main software components and software organization are discussed.
In Section~\ref{sec:l1fm} the pixel function manager is described.
In Section~\ref{sec:CodingPractices} coding practices are discussed.
In Sections~\ref{sec:ConfigurationDataManagement}, \ref{sec:ConfigurationDatabaseInterface}, \ref{sect:configobjects}, \ref{sec:ConfigurationDataUsage}, \ref{sec:configuration}, and \ref{sec:EventBuilderConfigurations} configuration is discussed.
In Sections~\ref{directorystructure}, \ref{sect:calib}, and \ref{sec:calibproc} the different calibrations are described.
In Sections~\ref{sec:dcs}, \ref{sec:hardware}, and \ref{sec:lowlevel} other stuff is described.
Other information about pixel online software is available at the pixel online software wiki pages~\cite{poswiki}.

